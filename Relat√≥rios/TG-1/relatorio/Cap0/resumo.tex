Algoritmos computacionais de busca t\^{e}m sido estudados por cientistas e companhias de engenharia desde o \'ultimo s\'eculo  devido �s aplica��es de tais algoritmos em situa��es reais, como agendamento de linhas a\'ereas, planejamento de rota em mapas, algoritmos de busca na Internet, planejamento de rotas em redes de computadores, rob�tica, etc. Usualmente, esses m�todos se baseiam em tipos abstratos de dados, como grafos e �rvores, para transferir um problema real para dentro de um contexto delimitado e intelig�vel. No campo da Ci\^{e}ncia da Computa\c{c}\~{a}o, grafos s�o tipos abstratos de dados que podem servir como suporte ferramental para algoritmos de busca.

De uma abordagem relacionada a algoritmos de busca em labirintos, este trabalho de gradua��o prop�e um m�todo para explorar grafos em um �mbito multiagente, em que os agentes n�o s�o capazes de se comunicarem entre eles, o que faz refer�ncia a problemas da realidade, como explora��o no fundo do mar, busca em estruturas compostas por muros maci�os, busca em locais in�spitos por agentes com restri��o energ�tica, etc.