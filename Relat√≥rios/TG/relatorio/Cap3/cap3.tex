\label{section_results}
We developed a code in Python based on the methods exposed in Section \ref{section_models_maze} for assessing our algorithm's performance. Pseudocode \ref{pseudocode_1} and Pseudocode \ref{pseudocode_2} guided the reasoning behind the code, that is available on \href{https://github.com/ArthurJose2000/TG}{github.com/ArthurJose2000/TG}.

This work used perfect mazes for simulating multi-agent scenarios where agents go through the mazes in order to find the goal. Section \ref{section_results_our_performance} presents the agents's performance when they use our algorithm. Section \ref{section_results_incremental_policy} proposes an agent policy modification when it finishes its interval for improving its performance. Finally Section \ref{section_results_tarry_vs_our} compares our algorithm's performance to the performance of the extended Tarry's algorithm \cite{KivelevitchCohen2010}, despite the latter having communication between agents.

\section{Our algorithm's perfomance}
\label{section_results_our_performance}

we worked with perfect mazes

\section{Incremental policy for the agents}
\label{section_results_incremental_policy}

\section{Our algorithm vs. Tarry's algorithm}
\label{section_results_tarry_vs_our}