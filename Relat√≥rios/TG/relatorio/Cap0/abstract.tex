Computational search algorithms have been studied by many scholars and engineering companies since the last century due to the real-life applications of such algorithms, such as in airline scheduling, planning path on maps, search engine algorithms, social media marketing, Internet routing protocols, robotics, etc. Usually, these methods rely on abstract data types, including graphs and trees, to transfer a real-world problem into a delimited and intelligible environment. In the context of Computer Science, graphs are abstract data types that might support computational search algorithms.

From the approach of maze-solving algorithms, this work aims to propose a method to explore a graph by coordinating agents that cannot communicate with each other. The agents are previously programmed to spread through the graph as dispersed as possible in order to avoid repeating the same portion of the graph and then minimizing the cost of steps to find the goal node. This approach is related to real problems, such as sea exploration, search in large wall structures, search with low energy-based agents in inhospitable environments, etc. Thus, there are previous motivations to do this research.

This thesis also investigates a mixed radix numerical representation as a viable tool, which, in the context of this work, is able to carry information both about values related to the visited nodes and their neighbors and the path's structure taken by an agent through the explored maze, which also gives pieces of evidence about the tree's structure, i.e., about the maze.

Finally, this report presents a comparison between our algorithm's performance and the performance of the extended Tarry's algorithm, proposed by \citen{KivelevitchCohen2010}, whose methods rely on communication between the agents, differently from this work's purpose, in which the agents cooperatively explore a graph with no communication.
