Computational search algorithms have been studied by many scholars and engineering companies since the last century due to the real-life applications of such algorithms, such as in airline scheduling, planning path on maps, search engine algorithms, social media marketing, Internet routing protocols, robotics, etc. Usually, these methods rely on abstract data types, including graphs and trees, to represent a real world problem as a well defined and computationally tractable algorithm. In the context of Computer Science, graphs are abstract data types that might support computational search algorithms.

Focusing on maze-solving algorithms, this work proposes a method to explore a graph by previously coordinated agents that cannot communicate with each other. The agents are previously programmed to spread through the graph as dispersed as possible in order to avoid traversing the same portion of the graph and then minimizing the number of steps to find the goal node. This approach is related to real problems, such as sea exploration, search in large wall structures, search with low energy-based agents in inhospitable environments, etc. Thus, there are previous motivations to do this research.

This work also investigates a mixed radix numerical representation as a tool, which, in the context of this research, is able to represent the visited nodes and their corresponding paths from the root, as well their related positions in an in-order traverse through the maze structured as a tree. It allows not only to continue on the path through its nearby neighbors but also to elaborate strategies to maximize the dispersion of the agents in the maze.

Finally, this report presents a comparison between our algorithm's performance and the performance of the extended Tarry's algorithm, proposed by \citen{KivelevitchCohen2010}, whose methods rely on communication between the agents, differently from this work's purpose, in which the agents cooperatively explore a graph with no communication.

It is worth emphasizing that, although the purpose of this work might be understood from the perspective of graphs in a general way, this research focuses on perfect mazes, i.e., trees.
