This work presented a solution for searching a goal node in a tree that is simultaneously visited by agents with no communication between them, where the agents are previously programmed to go through the tree as dispersed as possible. Despite the scope of this research being in the situation of search in undirected acyclic graphs, i.e., trees, it might be a kick-off to future works that aim to search a goal node in undirected cyclic graphs, and consequently to search the solution of imperfect mazes, in which there is more than one path to find the goal from any cell.

The approach of using a mixed radix numerical representation to the agent's path proved to be a powerful mathematical tool to represent, at the same numerical value, the agent's path and the convergence intervals of the nodes scanned by the agent. Moreover, it saves part of the tree's structure, which might be useful in other situations of graph exploration.

This work also could examine the performance of the extended Tarry's algorithm \cite{KivelevitchCohen2010}, which proved to be an effective algorithm to find maze solutions in a multi-agent context, even though it needs communication between the agents to run, differently from our algorithm.

It is also worth emphasizing that our algorithm doesn't have a strict acting, since it is open to improvements mainly when the agent finishes its starting interval. As seen in Section \ref{section_results_incremental_policy}, a simple policy increment produces a faster pioneer, which finds the goal with fewer steps. Therefore, more policies might be integrated above the basis of our algorithm, in order to make the agents smarter.

Thus, the authors proposed a multi-agent algorithm to search a goal node in trees, where the agents cannot communicate with each other. In this way, they need to be previously programmed to traverse the tree as dispersed as possible. Foremost, this research is open to improvements and comparisons with future works in the context of multi-agent graph exploration without communication.