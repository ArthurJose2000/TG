\section{Maze and multi-agent exploration}
\label{section_models_maze}
For simulation purposes, this work used an open-source tool that models a maze under a 4-neighbor 2D grid graph perspective, where each cell is a square with its edges composed by a wall or not. If there is a wall, an agent cannot traverse the maze across the related edge. On the other hand, if there is not a wall, an agent has a free way to traverse the maze through the related edge. It is worth to mention that, if an edge has a wall, the edge of the adjacent corresponding cell also has necessarily a wall. Likewise, if an edge has not a wall, the edge of the adjacent corresponding cell doesn't have a wall.

The maze has a goal that is a single marked cell, and an agent inside the maze aims to find the marked cell, traversing the maze cell by cell. This agent is an autonomous entity that follows a specific algorithm and previously programmed rules. So that it doesn't go through the same path more than one time, it stores the visited cells. Thus, when there are various posible branches to extend the agent's path, it ignores cells already visited by itself, and, if there is not a candidate to be a possible branch, the agent go back to the previous visited cell. Furthermore, if more than one agent traverses the maze in order to find the goal, differently from some approaches presented in \citen{Beisel2014}, \citen{Burgard2005}, and \citen{KivelevitchCohen2010}, there is no intercommunication between agents in the approach of this research, i.e., an agent knows nothing about another agent's search.

\citen{Muhammad2021} developed an open-source maze generator. It is a python module that creates randomly mazes and enables the user to simulate its own maze-solving algorithm. In this context, this work presents a multi-agent maze-solving algorithm that has been simulated over \citen{Muhammad2021} open-source code, with modifications. Figure \ref{maze_model} presents, for example, 2 agents traversing a $6 \times 6$ maze toward the goal.

\citen{Muhammad2021} software creates by default a ``perfect maze'', which means that there is one and only one path to the goal from any cell. However, it is possible to set the code to generate a imperfect maze.

\begin{figure}[ht!]
\centering
\includegraphics[width=0.5\textwidth]{Cap2/maze_model.png}
\caption{Blue and red agents traversing a $6\times 6$ maze. The goal is represented by the green entity. This maze is a perfect maze \cite{Muhammad2021}.}
\label{maze_model}
\end{figure}

\section{Maze from a graph topology perspective}
\label{section_models_maze_graph}
As pointed out in Section \ref{section_models_maze}, given that an agent ignores visited cells, there are important statements related to this work:

\begin{itemize}
\item if there is only one path to the goal from any cell, it is valid to consider a maze as a tree, i.e., a perfect maze \cite{Muhammad2021};

\item if there is more than one path to the goal from any cell, the maze cannot be considered as a tree, but it might be considered as a general 2D grid;

\item regarding the last statement, despite the maze cannot be considered as tree, an agent path might be individually interpreted as a tree if such agent ignores visited cells.
\end{itemize}

Figure \ref{maze_example_graph} gives an example of the graph representation of the maze presented in Figure \ref{maze_example}, where a red agent is traversing the maze toward the green goal. As established by \citen{Muhammad2021}, the maze cells are programmatically addressed as indices of a matrix, such as represented in Figure \ref{maze_indices_muhammad}.

\begin{figure}[ht!]
\centering
\includegraphics[width=0.5\textwidth]{Cap2/maze_indices_muhammad.png}
\caption{Maze cell indices. Source: \citen{Muhammad2021}.}
\label{maze_indices_muhammad}
\end{figure}

\begin{figure}
    \centering
    \subfloat[\centering Agent at starting cell]
    {{\includegraphics[width=3cm]{Cap2/maze_perfect_3x3_1agent_1.png} }}
    \qquad
    \subfloat[\centering Agent's 2nd step]
    {{\includegraphics[width=3cm]{Cap2/maze_perfect_3x3_1agent_2.png} }}
    \qquad
    \subfloat[\centering Agent's 10th step]
    {{\includegraphics[width=3cm]{Cap2/maze_perfect_3x3_1agent_3.png} }}
    \caption{Red agent traversing a maze toward the green goal. This maze is not a perfect maze.}
    \label{maze_example}
\end{figure}


\begin{figure}[ht!]
\centering
\includegraphics[width=0.5\textwidth]{Cap2/maze_example_graph.png}
\caption{Graph representation of the maze presented in Figure \ref{maze_example}. The cell indices are pointed out, and the cells visited by the red agent are also pointed out.}
\label{maze_example_graph}
\end{figure}	



\begin{forest}
for tree={delay={where content={}{content={\phantom{00}}}{}},s sep+=5mm,l+=5mm}
[1
  [00
     [00]
     []
     [10]
     []  
  ]
  []
  [10,edge={dashed,red,thick} 
     [00,edge={dashed,red,thick}
        [00]
        [01]
        [10]
        [11,edge={dashed,red,thick}]     
     ]
     []
     []
     [11]  
  ]
  [\phantom{00}]
]
\end{forest}

\begin{forest}
 [VP,circle,draw
 [DP][V’[V][DP]]
 ]
\end{forest}



\section{Multi-agent exploration without communication}
\label{section_models_exploration}
The goal of this work is to present a maze-solving algorithm in a multi-agent environment, where an agent cannot communicate with the other agents. Thus, each agent must be previously programmed to avoid exploring the same portion of the maze as other agents. Indeed, they should be as dispersed as possible.

First of all, this research considers a maze as a graph, where each node is a cell representation of the maze. Given a initial node that the agent starts its path through it, the agent checks if the node has children. Supposing that a node of the graph has at least one child, i.e., there are not completely walled cells, important statements are established:

\begin{itemize}
\item if an agent finds the node where is the goal, the agent finishes its path;

\item if an agent is not in the node where is the goal and the node has only one child, the agent necessarily goes through this node child;

\item if an agent is not in the node where is the goal and the node has more than one child, the agent needs to decide to which child it will move itself.
\end{itemize}

To establish an decision algorithm to the last statement, this work proposes firstly a interval division related to the agents and the children. Supposing that there are $k$ agents $a_{1}, a_{2},...,a_{k}$ to explore the maze, each agent will have a corresponding and proportional range of action related to the graph, as presented below:

\begin{equation}
	\begin{align}
			d = 1/k\\
		a_{1} \textnormal{: } [0, d[\\
		a_{2} \textnormal{: } [d, 2d[\\
		a_{3} \textnormal{: } [3d, 4d[\\
		...\\
		a_{k} \textnormal{: } [(k-1)d, 1]
	\end{align}
\end{equation}
where $d$ is the size of the partition corresponding to each agent. On the other hand, this work has handled convergence intervals to each node, where the agents tends to match its interval with the node convergence interval. In this way, an agent intends to traverse the maze as dispersed from the others as possible.

As pointed out in Section \ref{section_definitions_graph}, a graph may be represented as an adjacency list. Given that an agent path through an unknown maze may be represented as a tree, and considering the agent path as an adjacency list $A$, where the first explored node (root) $v_{1}$ has a convergence interval equal to $[0,1]$, while the other nodes $v_{i}$ of the maze are unknown, i.e., the agent doesn't know previously their convergence intervals, Pseudocode \ref{pseudocode_1} establishes the convergence intervals of each node. As mentioned previously, there is no communication between agents, therefore each agent needs to explore the maze individually, and calculates by itself the convergence intervals of each node. Furthermore, it is important to emphasize that each agent considers individually the maze as tree, where the root is the agent's starting node.

\begin{algorithm}
\caption{Traverse of the agent through the maze (interpreted as a tree by the agent).}
\label{pseudocode_1}
\begin{algorithmic}%[1]
\State /* \textit{Previously defined interval of the agent} */

\State agent\_interval $\gets$ $[a_{min},a_{max}[$

\State

\State /* \textit{Agent starts the exploration at node $v_{1}$} */

\State current\_node $\gets$ $v_{1}$

\State

\State /* \textit{Array of visited nodes by the agent is initially empty} */

\State visited\_nodes $\gets [$ $ ]$

\State

\State /* \textit{$v_{1}$ is the root, so its interval comprises the entire range of visits} */

\State intervals$[v_{1}] \gets [0,1]$

\State

\State /* \textit{Initialize the empty adjacency list of the tree $A$} */

\State $A \gets \oslash$

\State

\State /* \textit{In the tree $A$, create a node for $v_{1}$, and fill its children list with the neighboring cells of $v_{1}$. The children insertion order must follow some immutable rule. In the case of this work, the insertion is clockwise (North, East, South, West)} */

\State complete$(A, v_{1})$

\State

\State /* \textit{Create a flag to know if the agent filled its interval passing by all the correspondent nodes in the tree} */

\State finished\_interval $\gets$ FALSE

\State

\State

\While{current\_node \textit{is not the goal cell}}

\If{current\_node \textit{has no child} \textbf{or} \textit{all of} current\_node'\textit{s children were visited}}

\If{current\_node == $v_{1}$} \Comment{\textit{Unsuccessful search}}

\State \textbf{break} 

\Else

\State current\_node $\gets$ get\_parent(A, current\_node)

\State \textbf{continue}

\EndIf

\EndIf

\If{current\_node \textit{is not in} visited\_nodes}

\State append(visited\_nodes, current\_node)

\State max $\gets$ get\_max\_value(intervals[current\_node])

\State min $\gets$ get\_min\_value(intervals[current\_node]) \Comment{\textit{Node's mixed radix value}}

\State partition $\gets$ (max - min) \slash amount\_of\_children

\State count $\gets$ 0

\ForAll{child $v_{j}$ of current\_node in $A$}

\State intervals[$v_{j}$] $\gets$ [min + count $\cdot$  partition, min + (count+1) $\cdot$  partition[

\State complete(A, $v_{j}$)

\State count $\gets$ count + 1

\EndFor

\EndIf

\algstore{alg1}

\end{algorithmic}
\end{algorithm}

\begin{algorithm}
\ContinuedFloat
\caption{Traverse of the agent through the maze (interpreted as a tree by the agent).}
\begin{algorithmic}%[1]
\algrestore{alg1}

\ForAll{\textit{child} $v_{j}$ \textit{of} current\_node \textit{in} $A$}

\State max\_node $\gets$ get\_max\_value(interval[$v_{j}$])

\State min\_node $\gets$ get\_min\_value(interval[$v_{j}$])

\State max\_agent $\gets$ get\_max\_value(agent\_interval)

\State min\_agent $\gets$ get\_min\_value(agent\_interval)

\If{$v_{j}$ \textit{is in} visited\_nodes}

\State \textbf{continue}

\ElsIf{max\_agent < min\_node}

\State /* \textit{If the node's interval is on the right of the agent's interval, surely the}
\State \textit{agent has finished its interval. Since the order of the nodes are well defined,}
\State \textit{and in increasing order of intervals, the following condition improve the}
\State \textit{algorithm's performance. It is optional} */

\State finished\_interval $\gets$ TRUE

\State \textbf{break}

\ElsIf{min\_agent < max\_node \textbf{and} max\_agent > min\_node}

\State /* \textit{This condition verifies if there is intersection between the agent's interval}
\State \textit{and the node's interval. The order of children must be well defined} */

\State current\_node $\gets v_{j}$

\State \textbf{break}

\EndIf

\EndFor

\State

\State /* \textit{If} current\_node \textit{remains the same as it was at the beginning of the iteration,}

\State \textit{and it is the root, the agent has finished its interval} */

\If{current\_node == $v_{1}$ \textbf{and} finished\_interval == FALSE}

\State finished\_interval $\gets$ TRUE

\EndIf

\State

\State /* \textit{If the agent completely filled its interval, it needs to change its behavior, since its}
\State \textit{behavior is domain dependent. It will continue the DFS under some criteria.}
\State \textit{Otherwise, it must come back to parent in order to check if there are more nodes to}
\State \textit{visit} */

\If{finished\_interval == TRUE}

\State current\_node $\gets$ continue\_DFS\_under\_some\_criteria(A, current\_node)

\Else

\State current\_node $\gets$ get\_parent(A, current\_node)

\EndIf

\EndWhile

\end{algorithmic}
\end{algorithm}


Figure \ref{maze_example_graph_intervals} presents the converge intervals of the nodes if at least two agents explored the maze presented in Figure \ref{maze_example} using Pseudocode \ref{pseudocode_1}.

Thus, the goal of this work aims to establish an efficient algorithm that interrelates the agent intervals to the maze node intervals. Furthermore, the authors observed that a mixed radix representation to the agent path is a powerful mathematical tool to relate the path to the interval of the node that the agent is visiting, and it will be present in the next report.

\begin{figure}[ht!]
\centering
\includegraphics[width=0.5\textwidth]{Cap2/maze_example_graph_intervals.png}
\caption{Graph representation of the maze presented in Figure \ref{maze_example}, with intervals related to the agent's range of action.}
\label{maze_example_graph_intervals}
\end{figure}	

\section{Mixed radix representation to the agent path}
\label{section_models_mixed_radix}
In Section \ref{section_models_exploration}, this work presented a method for an agent to traverse the maze by defining ranges of action, despite it needs some improvements. Pseudocode \ref{pseudocode_1} explains this method, and establishes a way to explore the maze while an agent records the visited nodes and the explored tree. However, the agent doesn't need in fact to save the tree, because it only needs to know the current node interval to make a decision.

This work establishes a method to calculate the node intervals without the agent having necessarily to save the tree. In order to traverse the maze while the agent autonomously makes a decision node by node, it saves its path in a mixed radix numerical representation, where the starting representation is ``$0.$'' and it is limited to 1, i.e., $[0,1]$. According to the order of analysis of the children and the mathematical tools to calculate node intervals presented in Pseudocode \ref{pseudocode_1}, the items below present the agent behavior while it doesn't find the goal cell. It is important to emphasize that the value of the mixed radix numerical representation is the same as the minimum value of the node interval.

\begin{itemize}
\item if the current node has only one child, append $I_{1}$ to the mixed radix numerical representation of the path, and go to the child;

\item if the current node has $n$ children, and $n > 1$, then calculate the value of the mixed radix numerical representation of the path and, from this value, the related value to each child, and then make the decision according to the agent interval, as approached in Pseudocode \ref{pseudocode_1}. If the decision takes the \textit{i}th child, then append $(i-1)_{n}$ in the end of the agent path. Pseudocode \ref{pseudocode_2} presents a way to calculate computationally the value of the mixed radix numerical representation of the path;

\item if the current node has no child, go back to the parent, and remove the last appended value from the mixed radix numerical representation of the path.
\end{itemize}

Figure \ref{mixed_radix_path_example} presents an example for the mixed radix numerical representation approach. A blue agent with an interval $[1/3,2/3[$ is traversing a maze, and its current path is $0.0_{2}2_{3}$. The minimum value for each interval is also presented.

\begin{algorithm}
\caption{Methods related to mixed radix numerical representation approach.}
\label{pseudocode_2}
\begin{algorithmic}

\State /* \textit{Structure of mixed radix element} */

\State \textbf{struct} mr\_element

\State $\{$

\State \hspace{0.3cm} \textbf{type} numeral

\State \hspace{0.3cm} \textbf{type} radix

\State $\}$

\State

\State /* \textit{Initialize the agent's path. The agent is at the tree's root. Numerically it means zero in the mixed radix representation} */

\State $path \gets [$ $]$

\State

\Procedure{append}{path, numeral, radix}:

\State \textbf{struct} mr\_element element

\State element.numeral $\gets$ numeral

\State element.radix $\gets$ radix

\State add(path, element) \Comment{\textit{Append the element to the end of the} \textit{array} path}

\EndProcedure

\State

\Procedure{remove}{path}:

\State subtract(path)\Comment{\textit{Remove the last element of the} \textit{array} path}

\EndProcedure

\State

\Procedure{get\_mixed\_radix\_value}{path}:

\State value $\gets$ 0

\State interval\_size $\gets$ 1 \Comment{\textit{In order to calculate the interval size step by step}}

\ForAll {\textit{element in} path}

\If{element.numeral == I} \Comment{``I'' \textit{represents a node with only one child}}

\State \textbf{continue}

\EndIf

\State value $\gets$ value + interval\_size $\cdot$ (element.numeral / element.radix)

\State interval\_size $\gets$ interval\_size / element.radix

\EndFor

\State \textbf{return} value

\EndProcedure

\end{algorithmic}
\end{algorithm}}

\begin{figure}[ht!]
\centering
\includegraphics[width=0.5\textwidth]{Cap2/mixed_radix_path_example.png}
\caption{Blue agent with an interval $[1/3,2/6[$ traversing the maze presented in Figure \ref{maze_example}. Its current path is $0.0_{2}2_{3}$, that is equal to $2/6$.}
\label{mixed_radix_path_example}
\end{figure}

\section{Extended Tarry's algorithm}
\label{section_models_tarry_algorithm}